The key milestone this week will be to have a successful flight in a lawnmower pattern over an object-filled grid and to visualize the topographical map of the filled grid. The drone will be flown over the grid and once again, the controller's error will be checked. The controller will be tweaked until the RMSE values, flying over an object-filled grid, are within tolerance in the x and y position, and $\psi$, $\theta$, and $\phi$.\\
\indent The flight of the drone will be modified such that when the drone senses it's height dropping (sensing the leading edge of the object) it will stop moving in the x and y direction, and will hover up to correct the relative z-position. Once it has reached it's "relative" height of 1m, the drone will be commanded to move forward until it senses that it is too high (sensing the trailing edge of the object). Then the drone will stop moving in the x and y position and will move back down to an absolute z-position of 1 m. This serves as further controller error mitigation enabling the distinguishing of z-range measurement changes due to topography. The x and y positions will be determined as well and the topographic data points will then be used to visualize a contour map of the filled grid. 

