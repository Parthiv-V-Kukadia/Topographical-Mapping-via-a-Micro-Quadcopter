The key milestone this week will be to have a successful flight in a lawnmower pattern over an empty grid and to visualize the topographical map of the empty grid. A visual representation of the pattern can be seen in Fig. \ref{lawnmower}.  The drone begins at the bottom left corner which will be defined as the origin (0,0). The end of the grid in coordinate system is (1.5,1.5). The dark blue lines depict the path of the drone moving forward to the end of the grid. The red dashed line depicts the drone returning back to origin from the end.\\
\indent A successful flight is defined by having the 6 states of interest be within the tolerances. Specifically the goal of the flight test is to ensure the error in the x,y,z position and the roll, pitch, and yaw angles are within tolerance. This allows the controller error to be minimized. The values that will be used for tolerance are:
{\renewcommand\arraystretch{1.0}
\noindent\begin{longtable*}{@{}l @{\quad \le \quad} l@{}}
$O_x$ & 0.075 m\\
$O_y$ & 0.075 m\\
$O_z$ & 0.075 m\\
$\psi$ & 0.05 \\
$\theta$ & 0.015\\
$\phi$ & 0.015\\
\end{longtable*}}
Once the RMSE of the 6 states are within the tolerance limits, a topographical map of the empty grid will be created. The analysis of the data collected by the drone as well as the plotting of the map will be done using python.

