This paper will describe the design, implementation, and verification of a micro-quadcopter for topographical mapping. Topographic mapping has innumerable applications ranging from civilian hiking to military tactical positioning. The focus of this report is leveraging the CrazyFlie’s positioning capabilities for simple topographic mapping of a region of interest. This is motivated by the need to understand poorly mapped environments, such as the surface of different planets. This could enable successful landing of other spacecrafts, as well as facilitate further exploration of those regions. Implementation began with surveying a test area with the drone in a switch back pattern to ensure full coverage of the test section and collection of positional (x,y,z) information. To increase reliability, the drone was flown forward and backwards over the object filled grid, so that the data was collected twice. The geospatial data collected was then visualized in python producing an interactive three dimensional contour plot of the object filled grid. There were some challenges that had to be overcome, one of which was the inability of the custom observer and controller to fulfill flights within the defined RMSE values. Therefore the stock controller and observer of the CrazyFlie drone was used during the tests. The final visualizations were accurate to the real world object filled grid in terms of identifying objects, but not in dimensions. Topographical mapping via a micro-quadcopter provided a detailed visualization of the test grid, and can be applied in mapping larger surfaces using a larger quadcopter.