% This section must prepare the reader to understand the rest of your report. (What will you do? Why is it important? Who else has done similar things? How will your report be organized? To what, in particular, should readers pay attention?) This section must cite at least five references - for example, these can be final project reports that were written by students in prior years (Links to an external site.) (see below).

Topographical mapping provides a detailed record of a land area. The geographic positions and elevations of the region of interest as well as other geographic features can be determined by means of contour lines. A contour line is an imaginary line on the land surface connecting points of equal elevation. The goal of this project is to utilize the CrazyFlie's positioning capabilities to implement topographical mapping. This is motivated by the need to understand unmapped or poorly mapped environments. Topographic mapping by drone also enables easier access to otherwise inaccessible areas due the drone's versatility and ability to take off and fly just about anywhere. In addition, aerial mapping via drone is significantly faster than land based methods due to the labor intensive nature of geospatial data collection [1]. This project bears similarity to one done in Fall 2019 which focused on thermal mapping and another done in Fall 2014 which investigated sonar mapping [2,3]. Both projects produced contour maps of regions of interest. The former visualized the temperature gradient across and room while the latter plotted the intensity of audio signals over the test area. The current work also seeks to create a contour map over an area of interest, however the data being visualized will be topographic. 

The drone documents elevation data with a downward facing sensor underneath the CrazyFlie. This allows the drone to maintain a very precise z-position altitude by using the distance to the floor as absolute height [4]. This project takes advantage of the variable definition of "floor" to obtain altitude information about objects in the region of interest. The CrazyFlie maintains an absolute height above the floor directly below it. If the height of the floor were to suddenly change, i.e. placing a hand underneath the z-range sensor, the drone would readjust itself to maintain the same height above the hand or general obstacle. Through comparing the relative and absolute position of the drone the elevation of the object can be extracted. There are a multitude of use cases for topographic mapping ranging from recreational activities such as hiking to more technical disciplines such as designing any infrastructure project or flood prediction. 

In order to fully map out an area the drone will follow a lawnmower pattern over a region of specified size, collecting z-position data. This is similar to a project done in Fall 2014 which investigated coverage algorithms for a rectangular region of area [5]. The coverage method employed in this project will be the switchback method as termed in the 2014 project. This method was chosen due to its simplicity to implement, as opposed to the other method discussed in the 2014 project: random coverage mapping. There are marked advantages to the random coverage method, however the added implementation complexity as well the nature of topographical mapping dictate the switchback method to be the optimal choice. This flight path was also chosen in another project from Fall 2014 focused on wildlife surveillance to ensure full coverage of the area being surveyed [6]. It should be noted that the previous work maintained the drone at constant altitude throughout the aerial survey while the current investigation makes use of varying altitude. 

This report will provide the methodology for implementing topographic mapping utilizing the drone foundation developed in the labs. The report will first go through the controller and observer design and implementation followed by the methods employed to enable topographic mapping. Then the results of the flight tests will be visualized and discussed. Finally the big takeaways of the project will be presented as well as potential future work in topographical mapping.


