The final results can help draw several conclusions. Overall, the crazyflie drone's positioning capabilities using the flow deck were sufficient to create a three-dimensional topographical map of the grid that the micro-quadcopter flew over. The basic flight pattern used was of a switch back pattern, as shown in Fig. (\ref{fig:lawnmower}). There were many challenges that had to be overcome during the project. Firstly, the custom controller and observer caused the drone to swerve away from the desired flight path, due to which the stock controller and observer had to be used. Secondly, at higher velocities, the drone went haywire attempting to fly over an object filled grid, and for that reason, a slower forward-moving velocity had to be used. Thirdly, to ensure that there was full coverage of the 1.5 m x 1.5 m grid, the flight path had to be changed so that the drone would sweep around the edges of the grid to provide it more time to re-stabilize such that lower RMSE values were obtained. Lastly, the flight of the drone moving forward induced a sweep, where there was positive pitch induced on the drone, which resulted in the drone registering the (x,y) position with a delay, which was accounted for in python when using the geospatial data to visualize the topographical map.\\
\indent As seen from the experimental results, the drone was able to fly according to the desired flight path, with only some deviation. The drone's x position was slightly offset in the negative direction from the desired x-position, and the drone's y position was slightly offset to the right from the desired y-position. Overall, the drone's RMSE values in the x and y position were within tolerance. The topographical map produced using python showed the object-filled grid's surface visualization accurately, following the desired flight path that was provided to the drone. However, the dimensions of the box are not accurate, even though the locations and the fact that there is an object there is accurate. The drone was able to record the x and y position of these objects relatively well for lack of the states being observable, and the surface visualization represented the object-filled grid to a high degree of accuracy.\\
\indent However, this project can be improved upon in various ways if tested again in the future. One such improvement is detailed below. The flight of the drone could be modified such that when the drone senses it’s height dropping (sensing the leading edge of the object) it will stop moving in the x and y direction, and will hover up to correct the relative z-position. Once it has reached it’s "relative" height of 1m, the drone will be commanded to move forward until it senses that it is too high (sensing the trailing edge of the object). Then the drone will stop moving in the x and y position and will move back down to an absolute z-position of 1 m. This serves as further controller error mitigation enabling the distinguishing of z-range measurement changes due to topography. The x and y positions will be determined as well and the topographic data points will then be used to visualize a contour map of the filled grid.

Other improvements include implementing a custom controller and observer that was stable and observable would allow for a more stable flight with the drone being able to follow the desired flight path in a more accurate manner. Secondly, implementing a Loco-positioning system would increase the accuracy of results that are obtained, and it would allow mapping of more complex objects. However, the loco-positioning system also requires an extra deck, as well as anchors and tags, which increases the cost of the experiment, as well as the complexity of the project and it's implementation. Furthermore, if the project was able to be implemented on a surface that was not fixed, it would improve the application of the project - being able to track moving objects would allow the visualization of moving animals in forests or the movement of rocks/trees due to weather conditions, improving the extent of visualization provided. Lastly, being able to improve the stability of the drone would improve the application of this project through varying initial conditions not affecting the flight of the drone, and external factors not playing a role in the flight path of the drone. Overall, the project was a great success, and the results we got were accurate and what we expected, despite the challenges down the road.


